\documentclass{article}

\usepackage{times}
\usepackage{mathptmx}
\usepackage[margin=1in]{geometry}
\usepackage{graphicx}
\usepackage{hyperref}
\usepackage{xspace}

\title{Analysis of Behaviors of North American Animal Disease Simulation}
\author{Andrew J.\ Dolgert}
\date{5 July 2015}

\newcommand{\naadsm}{\textsc{naadsm}\xspace}
\newcommand{\hpai}{\textsc{hpai}\xspace}
\newcommand{\fmd}{\textsc{fmd}\xspace}

\begin{document}
\maketitle

\section{Introduction}
The North American Animal Disease Simulation (\naadsm) is an instrument to
understand the spatial spread of disease among agricultural units.
It is a primary tool of the United States Department of Agriculture
for developing policies regarding outbreaks of Foot and Mouth Disease
(\fmd) and High Pathogenicity Avian Influenza (\hpai).
The code for this simulation has been written by experts, both
veterinary and computational, and has been vetted numerous times.

At least four different sources describe what \naadsm models.
There are model specification documents with every detail but
the input data. The user guide for the graphical interface
elucidates choices about running the simulation. The code itself
is free on the web, and there are peer-reviewed articles about it.

We know the rules about small scale individual movements and behaviors that
were put into the model. We know that output from the simulation
looks like what experts expect at a larger scale.
However, the kinds of behaviors that shape policy happen at
an intermediate scale. This scale corresponds roughly to
what \naadsm calls a model within the simulation, for instance
the disease model, the direct contact model,
and the zone intervention model. What properties of agricultural
management and disease are most important for how disease spreads?
These are the traits upon which to focus interventions.

This document will offer ways to analyze \naadsm, as well
as other complex simulations, by giving tools to understand
intermediate-scale behaviors.

Analyzing the behavior of a model is a way to verify
functioning of a simulation according to its specification.


\section{Why This Simulation Is Difficult}

There is almost no data on how disease travels.
Even every detail of an outbreak is only one realization.
The particular strain of one outbreak differs from another,
as do local laws and management practices and landscapes
and hosts.

Without large-scale data, this simulation works from first principles.
Start with individual behavior and build up.
It's a game of telephone, lossy at every step.

The simulation serves inference, model selection,
and estimation. The job of inference is particularly difficult
because all of the data from an outbreak is a single trajectory,
at best.

Code encodes. It makes obscure original intentions.

\section{NAADSM is a Discrete Time Simulation}

Every simulation's ability to model the world depends
upon an internally consistent mathematical definition
of how state progresses with time.
ODEs, PDEs, continuous time, discrete time.

There is a clock.

Each possible transition has a probability mass function.
The transitions compete to fire.

Figure: Competition with probabilities of firing and four
ways to resolve it.

Four ways to handle competition among transitions to
change the same part of state. (Pull from other tex file.)

Consequence: All data in is, or will be treated as, a p.m.f.
of a hazard rate.

Consequence: A discrete-time version of infection pressure
exists, and this is what it is. We'll use it later.


\section{Techniques}
\begin{enumerate}
  \item Observe it like a wild trajectory, just with many more samples.
        This amounts to establishing observable states, then find every
        possible transition among observable states, then measure
        what you got. (This is a GSPN.)
  \item Establish simplified conditions, possibly which correspond
        to theoretical solutions.
  \item Turn parts of the simulation on and off in order to analyze them
        separately.
  \item Turn on developer logging in order to observe what caused an event.
  \item Write a simulation with which to compare.
  \item Lots of plots and graphs. These help you find the unexpected.
\end{enumerate}

\subsection{Unit Disease Model}
Figure: unit disease states and transitions as a graph.

Figure: unit disease pmf for each transition, from single-unit simulation.
        Include pmf that is input. Include continuous-time comparison.

Figure: unit disease pmf for each transition, from larger simulation.
        Include pmf that is input.


\subsection{Airborne Disease Spread Model}
Airborne spread, for the exponential model, is expressed
as the probability of infection, per day, at 1km.
The general form is $p=e^{-rd}$, where $d$ is a distance, $r$ a constant,
and $p$ the probability.
Given a distance of $d_0=1\:\mbox{km}$, the probability, $p_0$, is
$p_0 = e^{-r d_0},$
so that $r=-\ln p_0$. This becomes, succinctly, $p=p_0^d$, for
any given distance.

Instead of a probability per day, a continuous-time simulation would
pose a constant hazard rate. If we choose a hazard rate such that the
total probability of infection per day is $p$, then
\begin{equation}
  p=1-e^{-\lambda \Delta t},
\end{equation}
where $\lambda$ is the hazard rate and $\Delta t=1$ is the day. Thus,
$\lambda=-\ln(1-p)=-\ln(1-p_0^d)$.

Another approach is to take the probability per day as a geometric
distribution. Given that the distribution is enabled at time $t=0$,
there is a probability $p$ that it fires at day $t=1$, and so on.
The probability mass function is $(1-p)^k p$, where $k=t-1$ and $k$
runs from 0 upward. The mean of $k$ is  then $(1-p)/p$ so that the
mean of $t$ is $1/p$. The mean of an exponential distribution is
$1/\lambda$, so the equivalent distribution is $\lambda=p=p_0^d$.

Figure: hazard rates for curve given same start.

Figure: hazard rates from same initial condition, run just to first step
        repeatedly in order to measure hazard rate.


\subsection{Indirect Contact Spread Model}
Figure: Plots of spread from a single source, the same step over and over.

Figure: Cartoon of normalized trucks leaving versus normalized
        trucks arriving.

Figure: Force of infection for constant trucks leaving infected source
        versus constant trucks arriving to destinations.

Figure: Equivalence with simplified continuous-time models.


\subsection{Some Intervention Model: Zones?}


\section{Software Engineering}
Model readiness levels don't correspond to how we work with
simulations. They never survive contact with field data,
so the code, at this model level, is plastic.

People need GUIs, but statistics needs API and/or data formats.

Equations are always a challenge. (Pull from other document.)

Developer documentation and tracing are important.

\section{Conclusion}
\subsection{Going Forward}
Individual-based behaviors define a subspace within larger-scale
behaviors. Find and parameterize this.

\end{document}
