\documentclass{article}
\title{Approximation of Continuous Time with Discrete Time}
\author{Drew Dolgert}
\begin{document}
\maketitle

\section{Introduction}
Spatial spread of disease among agricultural locations
is an important problem.
Discrete stochastic simulation in continuous time provides
a precise methodology for modeling this spread.
Most simulation is done in discrete time, however,
because the problem size is large and
these can be faster to calculate.
How much of a difference does it make to use discrete-time
or continuous-time simulation?

There are many different ways to calculate the next state of
a discrete-time simulation. If the exact description of the
problem is a continuous-time model, then one exact
discrete-time model is defined by observing the continuous-time
model at fixed intervals.

There are two other known discrete-time dynamics of interest.
The \emph{conflict resolution method} records changes
to the state and resolves conflicts with rules whenever they
occur. The \emph{Molloy method} enumerates the complete
set of next states of the system and solves for their
likelihoods before sampling a new state. Both
methods will be described in greater detail in the following.

The final section describes a third method.
A well-posed continuous-time model predicts a set of
allowed trajectories, each with a likelihood from
which to sample the next state of the system.

\section{Conflict Resolution Method}
This is the simplest method, in use within \textsc{naadsm}.
Call the time of the system $T_n=n\Delta t$. The state of
the system at time $T_n$ is $X_n$.

Define the state of the system as a set of physical
states, $X_n=\{s_j\}$.
\begin{enumerate}
\item Given a state, $X_n$, and enabled transitions, $L_i$,
each transition produces a set of modified physical states,
$\{s_j^i\}$. The $\{s_j^i\}$ are not all states of the
system, just the modified ones.
\item If any of the same physical states in those sets aren't
equal, then apply a rule to decide a new state.
A rule looks at which transitions caused the state
change and what are the new states in order to choose
what new state to apply.
A rule is of the form $(s_j^i, s_j^k)\rightarrow s_j$.
\end{enumerate}

A conflict is defined according to whether the
final state of transitions disagrees. It's not a consideration
of probability in generation of the next state.


\section{Molloy Method}
This technique begins with an assumption that
the discrete time stochastic system is specified as a
set of discrete time stochastic processes, each of which
has an enabling time, $T^i_n=n\Delta t$, and a probability
mass function (pmf) $f_i(T-T^i_n)$. Transitions are in
conflict if they change the same starting state, as opposed
to resulting in different final states at $T_{n+1}$

Molloy's method begins by enumerating possible final
states at $T_{n+1}$. Given transition $a$ and $b$ in conflict,
the final states are $(!a,!b)$, $(!a,b)$, and $(a,!b)$,
with $(a,b)$ excluded.

The algorithm can be derived by imagining a two draws, one
for $a$ and one for $b$, where both are selected to occur.
Any time this happens, assign the result either to
$(!a,b)$ or $(a,!b)$, weighted by the relative probability
of $a$ and $b$. This generalizes to many transitions.


\section{Single Transition Method}
This section describes a way to create a discrete-time
model from a continuous-time model with the approximation
that there are no hidden intermediate states during the
time step.

Start with a continuous-time, discrete space model
for many processes.
It is possible for one transition's firing to disable
another transition but not the other way around.
Disabling of transitions can be expressed as a directed
graph.

If $a$ disables $b$, then $b$ can fire before $a$, but not
vice versa. The probability that $b$ fires before $a$ is
\begin{equation}
  \int_0^{t_b}d\tau_b\int_{t_b}^{\Delta t}d\tau_a
  f_b(\tau_b)f_a(\tau_a).
\end{equation}
This would compute the probability for transition to
a final state $(a,b)$ excluding $a$ before $b$.


\section{Enumeration of States}
What transitions may exclude the firing of other
transitions depends on the current state of the
system. Consider a continuous-time discrete-state
model model specified by stoichiometry. This makes
a matrix representation of states and transitions
practical.

The current state of the system is $\vec{m}$.
States and transitions form a bipartite graph
specified by two adjacency matrices, $A_{st}$ and
$A_{ts}$. Stochiomietry is associated with edges
of these graphs and represented by diagonal
matrices, $\mbox{diag}(S_i)$ and $\mbox{diag}(S_o)$,
for in and out of the transition.

From these matrices, we an construct a directed
graph of which transitions disable which others.




\end{document}
